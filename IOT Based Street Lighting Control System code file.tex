\documentclass[conference]{IEEEtran}
\IEEEoverridecommandlockouts
% The preceding line is only needed to identify funding in the first footnote. If that is unneeded, please comment it out.
\usepackage{cite}
\usepackage{amsmath,amssymb,amsfonts}
\usepackage{algorithmic}
\usepackage{graphicx}
\usepackage{textcomp}
\usepackage{xcolor}
\def\BibTeX{{\rm B\kern-.05em{\sc i\kern-.025em b}\kern-.08em
    T\kern-.1667em\lower.7ex\hbox{E}\kern-.125emX}}
\begin{document}

\title{IOT Based Street Lighting Control System\\
\thanks{Identify applicable funding agency here. If none, delete this.}
}

\author
{\IEEEauthorblockN{1\textsuperscript{st} MD. ZUNAYED ISLAM}
\IEEEauthorblockA{\textit{dept. Of CSE(Diploma} \\
\textit{ID: 200122127}\\
European University of Bangladesh}
\and
\IEEEauthorblockN{2\textsuperscript{st} FARVEJ HOSSAIN}
\IEEEauthorblockA{\textit{dept. Of CSE(Diploma} \\
\textit{ID: 2001221101}\\
European University of Bangladesh}
\and
\IEEEauthorblockN{3\textsuperscript{st} JARIN AKTER}
\IEEEauthorblockA{\textit{dept. Of CSE(Diploma} \\
\textit{ID: 200122141}\\
European University of Bangladesh}
\and
\IEEEauthorblockN{4\textsuperscript{st} TAHMINA SIAM RASHID}
\IEEEauthorblockA{\textit{dept. Of CSE(Diploma} \\
\textit{ID: 200122093}\\
European University of Bangladesh}
\and
\IEEEauthorblockN{5\textsuperscript{st} MD. FAHIM HUSSAIN}
\IEEEauthorblockA{\textit{dept. Of CSE(Diploma} \\
\textit{ID: 200122094}\\
European University of Bangladesh}
\and
\IEEEauthorblockN{6\textsuperscript{st} SURAIYA YESMIN}
\IEEEauthorblockA{\textit{dept. Of CSE(Diploma} \\
\textit{ID: 200122128}\\
European University of Bangladesh}
}

\maketitle

\begin{abstract}

Now-a-days the amount of power consumed by lighting and streets shares a major energy demand. Nowadays,
human has become too busy, and is unable to find time even to switch off the lights wherever not necessary.
Generally, street lights are switched on for whole night and during the day, they are switched off. But during
the night time, street lights are not necessary if there is no traffic. Saving of this energy is very important factor
these days as energy resources are getting reduced day by day. To overcome from this issue, a proper energy
saving methods and lighting control to be implemented. The proposed work is to have two controls like, one is
to switch of lights during no vehicle moments in streets and automatically switch it on when vehicles arrive
and the other modes are to give less intensity light for pedestrian and to switch on bright mode during vehicle
moments at sides on the roads. In this work the sensor is used for street arrangement, the Photo diodes and IR
sensors are used to sense vehicle moments. The control signals of sensors have been fed to microcontroller 8051.
Moreover the automatic and intelligent control schemes are required to control the complex lighting system
due to growth of cities and standard of living. 
\end{abstract}

\begin{IEEEkeywords}
Microcontroller, Streetlights, Photo Diodes, IR Sensor 
\end{IEEEkeywords}

\section{Introduction}
Generally, street light control system is a simple
concept which uses a transistor to turn ON in the
night time and turn OFF during the day time.
Providing street lighting is one of the most important
and expensive responsibilities of a city. Energy
efficient technologies and design mechanism can
reduce cost of the street lighting drastically. The work
related to design of energy efficient street lighting
system mainly focuses on using sensor based
technology. The entire process can be done by a using
a sensor namely LDR (Light dependent resistor).
Nowadays conserving the energy is an essential part
and day by day energy resources are getting decreased.
So our next generations may face a lot of problems
due to this lack of resources. This system doesn’t need
a manual operation to turn ON/OFF the street lights.
The street light system detects whether there is need
of light or not. This system was designed to detect the
light automatically and switch on's light. The design
makes use of a microcontroller to control the outputs
when it receives input from the resistor.
This design can be used in different areas like Street
lights, Public parks, and lights outside of houses. A
report was made to present an efficient street lighting
system with reduced power consumption in
comparison to other normal lighting systems by
knowing on this LED’s are more efficient than any
other diodes or bulbs.  

\section{RELATED WORK}

\subsection{Optimization of a standalone street light system
with consideration of lighting control. [1] }
This paper aims at designing and executing the
advanced development of embedded systems for
energy saving of street lamps. Nowadays, the human
has become too busy and is unable to find time even
to switch the lights wherever not necessary. This
paper gives the best solution for electrical power
wastage. Also, the manual operation of the lighting
system could eliminate. In this article, Light Emitting
Diode (LED) is used. In this system, the main
drawback was switching arrays of street lights were
not possible. Only Single Street can be operated.
\subsection{The efficient control algorithm for a smart solar
street light. [2] }
This proposed system works on solar energy. The
street light gets charged on sun energy in the daytime,
and it is consumed at night. The sensors get
automatically ON in the darkness and OFF in daytime.
When the battery of solar street light gets discharged,
it switched to RTC controller. If the weather changes,
the sun energy is not sufficient to charge the solar
batteries. Hence it may lead to the inconvenience of
the lighting of the street light. 

\subsection{E-Street zone-automatic Street light based on the
movement of vehicles. [3] }
Every street light can be integrated with an IR sensor
which detects the movement of vehicles. When the
vehicle passes, light gets illuminated. Due to this
electricity can be consumed less and energy can be
saved up to some extent. A solar panel has been
installed, and hence it gets charged by sunlight. But it
is impractical as street lights are also useful for the
people walking by the roadside and this sensor only
goes on when the vehicle passes it. Also, it is costly
due to IR sensors used in every single street light. 

\section{PROBLEM STATEMENT}
Lights contain chips. Chips consists Microcontroller
along with various sensors like CO2 sensor, fog sensor,
light intensity sensor, noise sensor and GSM modules
for wireless data sending and receiving between
concentrator and PC. The data from the chips would
get on a remote concentrator (PC), and the PC would
also transfer the controlling action to the chip.
According to the survey of variation in the intensity
of light in the field area, an efficient programming
would be done to ensure the least consumption of
energy. The emissions in the atmospheres would
detect along with the use of energy and any theft of
electricity.

\section{METHODOLOGY}


Automatic control of street lights is designed to turn
on and turn off street lights automatically. This
proposed system checks the amount of light. If light
is 80 percent available, it automatically turns off
street lights. But if amount of light is less than 80
percent, this project will automatically turn on street
lights. One can also adjust it according to its
requirement. Light sensor is used to detect intensity
of light. When LDR allows the current to flow the
block diagram of circuitry goes into working
condition. IR sensors start emitting IR rays via IR
transmitters. The IR diodes are placed on one side of
the road and photodiodes are placed on the other side of the road, directly facing the IR diodes. When
there is no vehicle on the highway. In this case, the
IR radiation emitted from the IR diode directly falls
on the photodiode which is exactly opposite to it.
This causes the photodiode to fall in conduction state.
This implies that photodiode conducts and current
passes through it. But as soon as any vehicle crosses
or obstructs the path of IR rays and prohibits it to
reach at IR receivers the microcontroller starts
getting the blockage signals. The programming
installed in microcontroller starts running which
basically presented here allows three street lights to
glow that are- the light in front of vehicle, behind
the vehicle and parallel to vehicle making backward
and forward street visible. Transformer converts the
high 230V AC to 12V AC. Rectifier converts it into
DC. For voltage regulation we are using LM 7805
and 7812 to produce ripple free 5 and 12 volts DC
constant supply. Emitting Diode (LED) replaces HID
lamps by engaging a programmable microcontroller
that controls the street light on/off conditions.

\subsection{Light Dependent Resistor}\label{AA}
A Light Dependent Resistor (LDR) is also called a
photo resistor or a cadmium sulphide (CdS) cell. It is
also called a photoconductor. It is basically a
photocell that works on the principle of
photoconductivity. The passive component is
basically a resistor whose resistance value decreases
when the intensity of light decreases.
This optoelectronic device is mostly used in light
varying sensor circuit, and light and dark activated
switching circuits. Some of its applications include
camera light meters, street lights, clock radios, light
beam alarms, reflective smoke alarms and outdoor
clocks.LDR used in street light automatically switches
ON when the night falls and turns OFF when the sun
rises. The snake like track shown is the Cadmium
Sulphide (CdS) film which also passes through the
sides. On the top and bottom are metal films which are
connected to the terminal leads. It is designed in such
a way as to provide maximum possible contact area
with the two metal films. The structure is housed in a
clear plastic or resin case, to provide free access to
external light. As explained above, the main
component for the construction of LDR is cadmium
sulphide (Cds), which is used as the photoconductor
and contains no or very few electrons when not
illuminated. In the absence of light it is designed to
have a high resistance in the range of mega ohms. As
soon as light falls on the sensor, the electrons are
liberated and the conductivity of the material
increases.

\subsection{Microcontroller}
A Microcontroller has all the necessary components
which a microprocessor possesses and invariably it
poses ROM, RAM, Serial Port, timers, interrupts
Input-Output ports, and a clock circuit. The
microcontroller always focus on the chip facility and
it is more prominent in the case of serial ports,
analog-todigital converters, timers, counters, readonly memory, parallel input, interrupt control,
random access memory, and output ports. The
concept of the 8051 microcontroller arises from here
and here we will discuss in depth about the various
aspects, uses, programming and other features of 8051
microcontrollers. 

\section{CONCLUSION}

This proposed system is a cost effective, practical, ecofriendly and the safest way to save energy. It clearly tackles the two problems that world is facing today,
saving of energy and also disposal of incandescent
lamps very efficiently. According to statistical data we
can save more that 40 % of electrical energy that is
now consumed by the highways. Initial cost and
maintenance can be the draw backs of this system.
But with the advancement in technology and good
resource planning the cost of the project can be cut
down and also with the use of good equipment the
maintenance can also be reduced in terms of periodic
checks.The LEDs have long life & emit cool light,
donor have any toxic material and can be used for fast
switching. For these reasons our system presents far
more advantages which can over shadow the present
limitations. Keeping in view the long term benefits
and the initial cost would never be a problem as the
investment return time is very less. The proposed
system has scope in various other applications like for
providing lighting in industries, campuses and
parking lots of huge shopping malls. This can also be
used for surveillance in corporate campuses and
industries.
\section{REFERENCES}

\begin{itemize}
\item Farah Ramadhani, and Kamalrulnizam Abu
Bakar, and Muham-Mad Gary Shafer.
“Optimisation of standalone street light system
withconsideration of lighting control”, IEEE,
2013.
\item Jain, Abhilasha, and Chandrasekhar Nagarajan.
"Efficient-Control Algorithm for a Smart Solar
Street Light." Next Generation Mobile
Applications, Services and Technologies, 2015.
\item P Arvind and V.Kishore . “E-Street ZoneAutomatic Streetlight based on the Movement
of Vehicles”, IJST 2016.

\item Energy efficient Smart Street Light, IEEE
conference 2017, Ravi kishorekodali and
Subbachary Yerroju, Department of Electronics
and Communication Engineering, National
Institute of Technology, Warangal. 
\item A smart street light intensity optimizer, IEEE
Conference 2017, Bilam Roy and Jayita Datta,
Department of Applied Electronics and
Instrumentation Engineering, Guru Nanak
Institute of Technology, Kolkata, W.B., India.
\item Internet of Things Based Intelligence Street
Lighting System for Smart city, IJIRSET, May
2016, Parkash, Prabhu V, Dandu Rajendra. 
\item Automatic Street Light Control System using
Wireless Sensor Networks, IEEE-2017. Dhiraj
Sunehra, SMIEEE, JNTU college of Engineering,
Department of Electronics and Communication
Engineering, Telangana, India. 
\end{document}
